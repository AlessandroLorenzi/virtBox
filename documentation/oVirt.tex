\documentclass[a4paper,10pt]{article}
\usepackage[utf8]{inputenc}
\usepackage{hyperref}

%opening
\title{oVirt - Gestione ottimale di virtual server}
\author{Alessandro Lorenzi}

\begin{document}
\maketitle

\textit{oVirt} é un sistema che fornisce un'intuitiva interfaccia web per la gestione multiutente
di macchine virtuali. Questo sistema é sviluppato e mantenuto da \href{www.it.redhat.com}{RedHat},
societá statiunitense dedita a sviluppo e supporto dei software open source. Come la stragrande
maggioranza dei progetti mantenuti da questa azienda anche \textit{oVirt}
é sotto \textit{licenza libera}, la \href{http://www.gnu.org/licenses/gpl-2.0.html}{GNU GPL 2}.

\section{Overview}
La struttura di \textit{oVirt} é composto da tre componenti principali.

Il primo componente é La macchina fisica su cui viene installata \href{http://fedoraproject.org}{fedora}.
Questa distribuzione permette di tutte le componenti di supporto per ospitare e gestire le macchine virtuali.\\
Il secondo componente é l'interfaccia web per host, utenti e vms.\\
Infine troviamo \href{http://freeipa.org/}{FreeIPA}, una soluzione che combina insieme Fedora Directory Server,
MIT Kerberos, NTP e DNS.

FreeIPA fornisce i servizi di autenticazione e autorizzazioni per l'intera applicazione. Viene inoltre fornita
l'immagine di un sistema ospite che al boot l'host carica le proprie credenziali kerberos dal sistema oVirt. 
Il sistema puó comunicare con ogni macchina ospite con connessione cifrata SASL utilizzando \textit{libvirt}.
Lo storage é garantito da \textit{iSCSI}.

\end{document}
